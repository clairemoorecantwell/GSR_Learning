\documentclass[12]{article}

\usepackage{tipa,gb4e}


\title{Manual: The Gradient Lexicon and Phonology Learner}
\author{Claire Moore-Cantwell}
\date{\today}


\begin{document}

\maketitle

\section{Quick Start}

\section{Details of Theories Implemented}
	\subsection{Preliminaries}
		\subsubsection{Theories of \textsc{Eval}}

		\subsubsection{Perceptron Learning}
	\subsection{UseListed}
		UseListed is a theory that many researchers implicitly or explicitly use as a default approach to exceptionality in phonology, but as far as I know no learning model has been developed.  The theory can easily be summarized as "We memorize exceptions."
		
		This theory assumes two things:
		\begin{enumerate}
			\item We memorize morphologically complex words, at least sometimes
			\item We can therefore produce morphologically complex forms in two ways:
			\begin{itemize}
				\item {\bf Composed} forms are created by accessing multiple morphemes and realizing them together according to the morphological and phonological grammar
				\item {\bf Listed} forms are accessed whole, and realized according to their lexical entry and the phonological grammar
			\end{itemize}
		\end{enumerate}
	
		A couple of examples: 
		
			In Tagalog, morphemes often undergo 'nasal substitution' in which the final nasal of a prefix coalesces with the initial consonant of the root, forming a single sound.
					In Tagalog, morphemes often undergo 'nasal substitution' in which the final nasal of a prefix coalesces with the initial consonant of the root, forming a single sound.
					
							In Tagalog, morphemes often undergo 'nasal substitution' in which the final nasal of a prefix coalesces with the initial consonant of the root, forming a single sound.
							
									In Tagalog, morphemes often undergo 'nasal substitution' in which the final nasal of a prefix coalesces with the initial consonant of the root, forming a single sound.
									
									
		In Tagalog, morphemes often undergo 'nasal substitution' in which the final nasal of a prefix coalesces with the initial consonant of the root, forming a single sound.
		
		
		\begin{exe}
			\ex
				\begin{xlist}
					\ex \gll {\bf d}in\'i\textscriptg  \hspace{3ex} /pa{\bf \textipa{N}}+{\bf d}in\'i\textscriptg/ \hspace{8ex} $\rightarrow$ pa{\bf n}-{\bf d}in\'i\textscriptg   \hspace{7ex} \textsc{Assimilation}\\
					{\it audible} {} {} {} {\it sense of hearing}\\
					\ex \gll {\bf d}al\'a\textipa{N}in \hspace{0.1ex} /i+pa{\bf \textipa{N}}+{\bf d}al\'a\textipa{N}in+in/  $\rightarrow$ \textglotstop i-pa-{\bf n}al\'a\textipa{N}in-in \hspace{0.5ex} \textsc{Substitution}\\
					{\it prayer} {} {} {} {\it to pray}\\
				\end{xlist}
			
		\end{exe}
		
		
		
		If we have both a {\bf Composed} form and a {\bf Listed} form available, there are many different ways we could decide between them.  
		
		

	\subsection{Lexically Indexed Constraints}

	\subsection{UR-constraints}

	\subsection{Representational Strength Theory}

	\subsection{Gradient Symbolic Representations}


\section{Supplementary details}
	\subsection{Dealing with hidden Structure}
	\subsection{Simulating \textsc{Gen}}
	\subsection{Applying constraints}

\section{Input file details}

\section{Classes and Methods}


\end{document}